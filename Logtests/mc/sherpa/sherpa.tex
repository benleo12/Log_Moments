\documentclass[10pt]{article}

\usepackage{fullpage}
\usepackage{amsmath}
\usepackage{helvet}
\usepackage{url}

\begin{document}

\title{{\Huge\bf Introduction to Sherpa}}
\author{{\Large Tutorial for summer schools}}
\date{}
\maketitle

\section{Introduction}

Sherpa is a complete Monte-Carlo event generator for particle production at
lepton-lepton, lepton-hadron, and hadron-hadron colliders~\cite{Gleisberg:2008ta}.
The simulation of higher-order perturbative QCD effects, including NLO corrections 
to hard processes and resummation as encoded in the parton shower, is emphasized 
in Sherpa. QED radiation, underlying events, hadronization and hadron decays can
also be simulated. Alternatively, Sherpa can be used for pure parton-level NLO QCD 
calculations with massless or massive partons.

Many reactions at the LHC suffer from large higher-order QCD corrections.
The correct simulation of Bremsstrahlung in these processes is essential. It can 
be tackled either in the parton-shower approach, or using fixed-order calculations.
Sherpa combines both these methods using a technique known as Matrix Element + 
Parton Shower merging (ME+PS). Details are described in Ref.~\cite{Buckley:2011ms}
and have been discussed in the lectures. This tutorial will show
you how to use the method in Sherpa.

The Docker container used in this tutorial can be obtained by running
\begin{verbatim}
  docker pull sherpamc/sherpa:2.2.7
\end{verbatim}
Should you have problems with disk space, consider running
{\tt docker containers prune} and {\tt docker system prune}.
Sherpa is installed in {\tt /usr/local}, and its documentation is found
online~\cite{SherpaManual}. Example setups are located in {\tt /usr/local/Sherpa}.
To run the docker container, use the following command
\begin{verbatim}
  docker run -it -u $(id -u $USER) --rm -v $HOME:$HOME -w $PWD sherpamc/sherpa:2.2.7
\end{verbatim}
You may simplify this by defining an alias. We also add an environment variable
needed for Rivet analyses
\begin{verbatim}
  alias docker-run-sherpa='docker run -it -u $(id -u $USER) \
    --rm -v $HOME:$HOME -w $PWD --env="RIVET_ANALYSIS_PATH=." sherpamc/sherpa:2.2.7'
\end{verbatim}
The tutorial can be obtained by running
\begin{verbatim}
  git clone https://gitlab.com/shoeche/tutorials.git
\end{verbatim}
All paths referenced to in these instructions are relative to the {\tt tutorials/}
directory created in this step. We will use a dedicated Rivet analysis, which is made
available in the directory {\tt rivet/}. As a first step, please compile this analysis
and link the resulting library to your run directory using the sequence of commands
\begin{verbatim}
  cd rivet/
  docker-run-sherpa rivet-buildplugin MC_TTBar.cc
  cp RivetAnalysis.so ../mc/sherpa/
\end{verbatim}


\subsection{The Input File}

Sherpa is steered using input files, which consist of several sections.
A comprehensive list of all input parameters for Sherpa is given in the
Sherpa manual~\cite{SherpaManual}. For the purpose of this tutorial, 
we will focus on the most relevant ones.

Change to the directory {\tt tutorials/mc/sherpa} and open the file {\tt Run.dat}
in an editor. Have a look at the section which is delimited by the tags {\tt(run)\{}
and {\tt \}(run)} (We will call this section the {\tt(run)} section in the following). 
You will find the specification of the collider, i.e.\ its beam type and center-of-mass 
energy, as well as a couple of other parameters, which will be explained later.

The {\tt(processes)} section specifies, which reactions are going to be simulated.
Particles are identified by their PDG codes, i.e.\ 1 stands for a down-quark, 
-1 stands for an anti-down, 2 for an up-quark, etc. 
The special code 93 represents a ``container'', which comprises all light quarks,
b-quarks, and the gluon. It is also called the ``jet'' container.

\subsection{Running Sherpa}
\label{sec:running_sherpa}

Have you found out which physics process is going to be generated?
You can verify your guess by running the command line (Note that all commands
in this section must be prefixed by {\tt docker-run-sherpa} or run inside a shell
started by {\tt docker-run-sherpa bash}. Running inside a shell is recommended.
See the introduction for details.)
\begin{verbatim}
  Sherpa -e1 -o3
\end{verbatim}
When run for the first time, Sherpa will produce diagram information for the calculation
of hard scattering processes. It will also compute that hard scattering cross sections,
which are stored, together with the diagram information, for subsequent runs.

The option {\tt -e1} used above instructs Sherpa to produce one event, and the option 
{\tt -o3} will print the result of event generation on the screen. You will see Sherpa's 
internal event record. Search for {\tt Signal Process} inside the output and check 
incoming and outgoing particles.
What happens to the unstable particles after they have been produced in the hard 
scattering? Is this result physically meaningful?

Have a look at the Feynman diagrams, which contribute to the simulation of the hard process:
\begin{verbatim}
  plot_graphs.sh graphs/
  firefox graphs/index.html
\end{verbatim}
Are these the diagrams you expect to find? If not, which ones are missing?
Can you find the setting in the runcard which restricts the set of diagrams?

\subsection{Unstable particles}

Open the input file again. Add the setting {\tt HARD\_DECAYS On;}
to the {\tt(run)} section, which instructs Sherpa to automatically 
decay unstable particles. Verify that the particles you wish to decay are
flagged unstable by checking the screen output of Sherpa during runtime. 
Search for the {\tt `List of Particle Data'}. Note that you can set particles
stable individually using the switch {\tt STABLE[<PDG ID>]=1}.

For experts: What happens when you change or remove the setting {\tt WIDTH[6] 0;}
(Hint: Search for {\tt `Performing tests'} in the screen output of Sherpa).
Can you guess what the problem might be?

\subsection{ME+PS merging}

The current runcard lets Sherpa generate events at lowest order in the strong coupling.
To improve the description of real radiative corrections, we can include higher-multiplicity
tree-level contributions in the simulation. This is done by changing the process specification

\begin{verbatim}
  Process 93 93 -> 6 -6;
\end{verbatim}
to
\begin{verbatim}
  Process 93 93 -> 6 -6 93{1};
\end{verbatim}
The last entry instructs Sherpa to produce up to one additional ``jet'' using
hard matrix elements and combine the respective process with the leading-order process.
This is known as Matrix Element + Parton Shower merging (ME+PS), or the CKKW method.
The essence of the method is a separation of hard from soft radiative corrections,
achieved using phase-space slicing by means of a variable called the jet criterion. The slicing parameter is called the merging cut.

Let us assume we want to classify jets of more than 20~GeV transverse momentum as hard.
In Sherpa, the corresponding merging cut would be specified as
\begin{verbatim}
  CKKW sqr(20/E_CMS);
\end{verbatim}
Therefore, the complete {\tt (processes)} section for the merged event sample reads:
\begin{verbatim}
  (processes){
    Process 93 93 -> 6 -6 93{1};
    CKKW sqr(20/E_CMS);
    Order (*,0);
    End process;
  }(processes);
\end{verbatim}
If you like, you can have a look at the Feynman graphs again, which contribute
to the ME+PS merged event sample. In this case, you should not remove the line
{\tt Print\_Graphs graphs;} from the {\tt (processes)} section, and rerun the 
plot command from Sec.~\ref{sec:running_sherpa}.

Run the new setup. Why does Sherpa compute another cross section?

\subsection{Analyses}

By default, Sherpa does not store events. Run Sherpa with the following command 
to write out weighted event files in HepMC format, which can subsequently be analyzed
\begin{verbatim}
  Sherpa EVENT_OUTPUT=HepMC_Short[events_1j] -wW
\end{verbatim}
Sherpa will produce a gzipped file called {\tt events\_1j.hepmc}, 
which can be processed with Rivet using the command
\begin{verbatim}
  mkfifo events
  gunzip -c events_1j.hepmc > events & \
  rivet -a MC_TTBar -H analysis_1j.yoda events
\end{verbatim}
The option {\tt -a MC\_TTBar} instructs Rivet to run a Monte-Carlo analysis 
of semileptonic $t\bar{t}$ events, which will provide us with a few observables
that can be used for diagnostic purposes.
Using a fifo pipe to extract the compressed event file avoids having to expand
this file on disk, which can take substantial time and would unnecessary
take up storage space. This method will prove very useful in the tutorial
on boosted final states.


You can view the results of this analysis by running the command
\begin{verbatim}
  rivet-mkhtml --mc-errs analysis_1j.yoda
\end{verbatim}
and opening {\tt rivet-plots/index.html} in a browser.

Now it is your turn: Generate a separate event file without ME+PS merging 
and analyze it with Rivet. Compare the results to your previous analysis 
using {\tt rivet-mkhtml} (Hint: You can specify multiple yoda input files 
to {\tt rivet-mkhtml}.) Do you observe any differences? Why?

\subsection{Playing with the setup}

Here are a few suggestions to try if you have time left during the tutorial.

\subsubsection{Changing the functional form of scales}

You may have noticed the following line in the runcard
\begin{verbatim}
  CORE_SCALE VAR{sqr(175)};
\end{verbatim}
It invokes Sherpa's interpreter to compute the renormalization and factorization
scale for the $pp\to t\bar{t}$ production process as the mass of the top quark
(Note that all scales have dimension GeV$^2$, hence the scale is squared using
the {\tt sqr} function).

You can change this to a different value. For example, you could use
the invariant mass of the top quark pair:
\begin{verbatim}
  CORE_SCALE VAR{Abs2(p[2]+p[3])};
\end{verbatim}
Note that when you change the scale, the cross section will change and needs to be recomputed!
You can instruct Sherpa not to pick up the old result and compute a new one by launching 
it with the commandline option {\tt -r Result\_NewScale/}. New cross section information
will then be stored in the directory {\tt Results\_NewScale/}, while the old is still present
in the directory {\tt Results/}.

How do the observables change with the new scale and why?
What is the effect of ME+PS merging when using this scale?

\subsubsection{Hadronization and underlying event}

So far, multiple parton interactions, which provide a model for the underlying event,
have not been simulated. Also, hadronization has not been included in order to speed up
event generation. You can enable both by removing or commenting the line
\begin{verbatim}
  MI_HANDLER None; FRAGMENTATION Off;
\end{verbatim}
How do the predictions for the observables change and why?

\subsubsection{Variation of the merging cut}

ME+PS merging is based on phase-space slicing, and the slicing cut is called 
the merging scale. Is is an unphysical parameter, and no observable should
depend sizeably on its precise value. Verify this by varying the merging
cut by a factor of two up and down.

Note that the initial cross sections that Sherpa computes will be different 
for different values of the merging cut (Why?). You should therefore instruct 
Sherpa to use different result directories for each run in the test. 
The result directory is specified on the command line with option {\tt -r},
for example
\begin{verbatim}
  Sherpa -r MyResultDirectory/
\end{verbatim}

\subsubsection{Other hard scattering processes}

Try to generate other hard scattering processes, like the production
of a Drell-Yan muon pair (Hint: The PDG id for the muon is {\tt 13}.)
You will have to insert an additional section into the runcard, which
reads
\begin{verbatim}
  (selector){
    Mass 13 -13 66 116;
  }(selector);
\end{verbatim}
Why is this section needed?

\section{Uncertainty estimates}

You may ask yourself what the meaningful variations are to assess
the uncertainty of your generator prediction. We will try to answer this
question in the next part of the tutorial. As it can be quite costly 
to perform a complete uncertainty study, you will work with your peers 
and split the workload.

Using Sherpa, you will study the renormalization and factorization
scale dependence of the reaction $pp\to t\bar{t}$ as well as the uncertainty
related to the PDF fit. This is done on-the-fly, using reweighting techniques
that are similar to the ones you implemented in the tutorial on parton showers.
The reweighting is enabled in Sherpa by adding the following flags to the
{\tt (run)} section of you input file:
\begin{verbatim}
  SCALE_VARIATIONS 0.25,0.25 0.25,1 1,0.25 1,4 4,1 4,4;
  PDF_LIBRARY LHAPDFSherpa; PDF_SET CT14nlo;
  PDF_VARIATIONS CT14nlo[all];
\end{verbatim}
The line starting with {\tt SCALE\_VARIATIONS} instructs Sherpa to perform a
six-point variation of the renormalization and factorization scales,
with the pairs of numbers representing the scale factors for $\mu_R^2$
and $\mu_F^2$. The line starting with {\tt PDF\_VARIATIONS} instructs Sherpa
to perform a variation using all PDFs in the error set {\tt CT14nlo}, which
is interfaced through LHAPDF. Finally, the default PDF is set to {\tt CT14nlo}.

We will not let Sherpa write event files in this part of the tutorial,
but use the internal interface to Rivet, as it allows us to run the analysis
simultaneously for all scale and PDF variations. This is set up by adding
the following lines to your runcard:
\begin{verbatim}
(analysis){
  BEGIN_RIVET {
    -a MC_TTBar;
  } END_RIVET;
}(analysis);
\end{verbatim}
We need to instruct Sherpa to enable the internal Rivet interface using
\begin{verbatim}
  Sherpa -wW -aRivet -AAnalysis/A
\end{verbatim}
If you have enabled the usage of multiple CPUs or multiple CPU cores on your VM
you may want to run Sherpa in MPI mode to accelerate the event generation:
\begin{verbatim}
  mpirun -n 2 Sherpa -wW -aRivet -AAnalysis/A
\end{verbatim}
To improve the statistical significance of our simulation, we can simulate
only the relevant decays of the $W^\pm$-bosons. To simplify things, we will
ignore decays into $\tau$ and choose hadronic modes for the $W^-$ and 
leptonic modes for the $W^+$:
\begin{verbatim}
  HDH_STATUS[-24,-12,11] 0;
  HDH_STATUS[-24,-14,13] 0;
  HDH_STATUS[-24,-16,15] 0;
  HDH_STATUS[24,16,-15] 0;
  HDH_STATUS[24,2,-1] 0;
  HDH_STATUS[24,4,-3] 0;
\end{verbatim}
You can use the plot script provided in the top level directory to merge and
display the results of the simulation
\begin{verbatim}
  cd ..
  ./plotit.sh sh
\end{verbatim}

\begin{thebibliography}{1}

\bibitem{Gleisberg:2008ta}
T.~Gleisberg et~al.
``Event generation with SHERPA 1.1.''
JHEP {\bf 02} (2009) 007.

\bibitem{Buckley:2011ms}
A.~Buckley et~al.
``General-purpose event generators for LHC physics.''
Phys.\ Rept.\ {\bf 504} (2011) 145.

\bibitem{SherpaManual}
https://sherpa.hepforge.org/doc/SHERPA-MC-2.0.0.html.

\end{thebibliography}

\end{document}
